%%%%%%%%%%%%%%%%%%%%%%%%%%%%%%%%%
% PACKAGE IMPORTS
%%%%%%%%%%%%%%%%%%%%%%%%%%%%%%%%%
\usepackage[fontset=adobe]{ctex}
\usepackage[tmargin=2cm,rmargin=1in,lmargin=1in,margin=0.85in,bmargin=2cm,footskip=.2in]{geometry}
\usepackage{fontspec} % 支持自定义字体
\usepackage{float}
\usepackage{graphicx}
\usepackage{listings}
\lstset{
    language=Python,
    basicstyle=\small\ttfamily,
    keywordstyle=\color{blue},
    stringstyle=\color{red},
    commentstyle=\color{green},
    showstringspaces=false,
    breaklines=true,   % 启用自动换行
    breakatwhitespace=false,  % 是否仅在空白字符处换行
}
\usepackage{tocloft}
% 引入hyperref包并设置超链接颜色
\usepackage{hyperref}
\hypersetup{
    colorlinks=true,       % 颜色链接而不是方框
    linkcolor=black,        % 目录链接颜色
    citecolor=blue,        % 引用的颜色
    filecolor=blue,        % 文件链接颜色
    urlcolor=blue          % URL 链接颜色
}

\usepackage{fancyhdr}
\pagestyle{fancy}
% 默认页眉和页脚设置
\fancyhf{}
% 设置页眉
\fancyhead[LE]{} % 左侧偶数页(左侧页眉)
\fancyhead[CE]{} % 中间偶数页(页码)
\fancyhead[RE]{} % 右侧偶数页(右侧页眉)
\fancyhead[LO]{} % 左侧奇数页(左侧页眉)
\fancyhead[CO]{} % 中间奇数页(页码)
\fancyhead[RO]{\leftmark} % 右侧奇数页(右侧页眉)

% 设置页脚
\fancyfoot[LE]{\thepage} % 左侧偶数页(左侧页脚)
\fancyfoot[CE]{} % 中间偶数页(页码)
\fancyfoot[RE]{} % 右侧偶数页(右侧页脚)
\fancyfoot[LO]{} % 左侧奇数页(左侧页脚)
\fancyfoot[CO]{} % 中间奇数页(页码)
\fancyfoot[RO]{\thepage} % 右侧奇数页(右侧页脚)

\setlength{\headheight}{15pt}
\setlength{\footskip}{15pt}
%%%%%%%%%%%%%%%%%%%%%%%%%%%%%%%%%
% FONTSET
%%%%%%%%%%%%%%%%%%%%%%%%%%%%%%%%%

%%%%%%%%%%%%%%%%%%%%%%%%%%%%%
% 使用电脑内字体
%%%%%%%%%%%%%%%%%%%%%%%%%%%%%
\setmonofont{Ubuntu Mono}    % 等宽字体

% \setmainfont{TimesNewRoman}[
%     Path=./latex-fonts/TimesNewRoman/,
%     BoldFont=TimesNewRoman-Bold
% ] % 英文字体
% \setsansfont{NotoSansSC-Regular}[
%     Path=./latex-fonts/NotoSansSC/
% ] % 无衬线字体
% \setmonofont{Consola}[Path=./latex-fonts/Consolas/, Extension=.ttf]     % 等宽字体
%%%%%%%%%%%%%%%%%%%%%%%%%%%%%%%%%
% CJK FONTSET
%%%%%%%%%%%%%%%%%%%%%%%%%%%%%%%%%
% \setCJKmainfont{NotoSerifSC-Regular}[
%     Path=./latex-fonts/NotoSerifSC/,
%     Extension=.otf,
%     ItalicFont=NotoSerifSC-Light,
%     BoldFont=NotoSerifSC-Bold
% ]
% \setCJKmonofont{Consola}[Path=./latex-fonts/Consolas/, Extension=.ttf]


% hyphenpenalty值越大断字出现的就越少
\hyphenpenalty=5000
% tolerance越大,换行就会越少
% 也就是说,LaTex会把本该断开放到下一行的单词,整个儿的留在当前行
\tolerance=1000

